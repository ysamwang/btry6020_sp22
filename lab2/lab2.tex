% Options for packages loaded elsewhere
\PassOptionsToPackage{unicode}{hyperref}
\PassOptionsToPackage{hyphens}{url}
%
\documentclass[
]{article}
\title{Lab 2}
\author{Y. Samuel Wang}
\date{2/4/2022}

\usepackage{amsmath,amssymb}
\usepackage{lmodern}
\usepackage{iftex}
\ifPDFTeX
  \usepackage[T1]{fontenc}
  \usepackage[utf8]{inputenc}
  \usepackage{textcomp} % provide euro and other symbols
\else % if luatex or xetex
  \usepackage{unicode-math}
  \defaultfontfeatures{Scale=MatchLowercase}
  \defaultfontfeatures[\rmfamily]{Ligatures=TeX,Scale=1}
\fi
% Use upquote if available, for straight quotes in verbatim environments
\IfFileExists{upquote.sty}{\usepackage{upquote}}{}
\IfFileExists{microtype.sty}{% use microtype if available
  \usepackage[]{microtype}
  \UseMicrotypeSet[protrusion]{basicmath} % disable protrusion for tt fonts
}{}
\makeatletter
\@ifundefined{KOMAClassName}{% if non-KOMA class
  \IfFileExists{parskip.sty}{%
    \usepackage{parskip}
  }{% else
    \setlength{\parindent}{0pt}
    \setlength{\parskip}{6pt plus 2pt minus 1pt}}
}{% if KOMA class
  \KOMAoptions{parskip=half}}
\makeatother
\usepackage{xcolor}
\IfFileExists{xurl.sty}{\usepackage{xurl}}{} % add URL line breaks if available
\IfFileExists{bookmark.sty}{\usepackage{bookmark}}{\usepackage{hyperref}}
\hypersetup{
  pdftitle={Lab 2},
  pdfauthor={Y. Samuel Wang},
  hidelinks,
  pdfcreator={LaTeX via pandoc}}
\urlstyle{same} % disable monospaced font for URLs
\usepackage[margin=1in]{geometry}
\usepackage{color}
\usepackage{fancyvrb}
\newcommand{\VerbBar}{|}
\newcommand{\VERB}{\Verb[commandchars=\\\{\}]}
\DefineVerbatimEnvironment{Highlighting}{Verbatim}{commandchars=\\\{\}}
% Add ',fontsize=\small' for more characters per line
\usepackage{framed}
\definecolor{shadecolor}{RGB}{248,248,248}
\newenvironment{Shaded}{\begin{snugshade}}{\end{snugshade}}
\newcommand{\AlertTok}[1]{\textcolor[rgb]{0.94,0.16,0.16}{#1}}
\newcommand{\AnnotationTok}[1]{\textcolor[rgb]{0.56,0.35,0.01}{\textbf{\textit{#1}}}}
\newcommand{\AttributeTok}[1]{\textcolor[rgb]{0.77,0.63,0.00}{#1}}
\newcommand{\BaseNTok}[1]{\textcolor[rgb]{0.00,0.00,0.81}{#1}}
\newcommand{\BuiltInTok}[1]{#1}
\newcommand{\CharTok}[1]{\textcolor[rgb]{0.31,0.60,0.02}{#1}}
\newcommand{\CommentTok}[1]{\textcolor[rgb]{0.56,0.35,0.01}{\textit{#1}}}
\newcommand{\CommentVarTok}[1]{\textcolor[rgb]{0.56,0.35,0.01}{\textbf{\textit{#1}}}}
\newcommand{\ConstantTok}[1]{\textcolor[rgb]{0.00,0.00,0.00}{#1}}
\newcommand{\ControlFlowTok}[1]{\textcolor[rgb]{0.13,0.29,0.53}{\textbf{#1}}}
\newcommand{\DataTypeTok}[1]{\textcolor[rgb]{0.13,0.29,0.53}{#1}}
\newcommand{\DecValTok}[1]{\textcolor[rgb]{0.00,0.00,0.81}{#1}}
\newcommand{\DocumentationTok}[1]{\textcolor[rgb]{0.56,0.35,0.01}{\textbf{\textit{#1}}}}
\newcommand{\ErrorTok}[1]{\textcolor[rgb]{0.64,0.00,0.00}{\textbf{#1}}}
\newcommand{\ExtensionTok}[1]{#1}
\newcommand{\FloatTok}[1]{\textcolor[rgb]{0.00,0.00,0.81}{#1}}
\newcommand{\FunctionTok}[1]{\textcolor[rgb]{0.00,0.00,0.00}{#1}}
\newcommand{\ImportTok}[1]{#1}
\newcommand{\InformationTok}[1]{\textcolor[rgb]{0.56,0.35,0.01}{\textbf{\textit{#1}}}}
\newcommand{\KeywordTok}[1]{\textcolor[rgb]{0.13,0.29,0.53}{\textbf{#1}}}
\newcommand{\NormalTok}[1]{#1}
\newcommand{\OperatorTok}[1]{\textcolor[rgb]{0.81,0.36,0.00}{\textbf{#1}}}
\newcommand{\OtherTok}[1]{\textcolor[rgb]{0.56,0.35,0.01}{#1}}
\newcommand{\PreprocessorTok}[1]{\textcolor[rgb]{0.56,0.35,0.01}{\textit{#1}}}
\newcommand{\RegionMarkerTok}[1]{#1}
\newcommand{\SpecialCharTok}[1]{\textcolor[rgb]{0.00,0.00,0.00}{#1}}
\newcommand{\SpecialStringTok}[1]{\textcolor[rgb]{0.31,0.60,0.02}{#1}}
\newcommand{\StringTok}[1]{\textcolor[rgb]{0.31,0.60,0.02}{#1}}
\newcommand{\VariableTok}[1]{\textcolor[rgb]{0.00,0.00,0.00}{#1}}
\newcommand{\VerbatimStringTok}[1]{\textcolor[rgb]{0.31,0.60,0.02}{#1}}
\newcommand{\WarningTok}[1]{\textcolor[rgb]{0.56,0.35,0.01}{\textbf{\textit{#1}}}}
\usepackage{graphicx}
\makeatletter
\def\maxwidth{\ifdim\Gin@nat@width>\linewidth\linewidth\else\Gin@nat@width\fi}
\def\maxheight{\ifdim\Gin@nat@height>\textheight\textheight\else\Gin@nat@height\fi}
\makeatother
% Scale images if necessary, so that they will not overflow the page
% margins by default, and it is still possible to overwrite the defaults
% using explicit options in \includegraphics[width, height, ...]{}
\setkeys{Gin}{width=\maxwidth,height=\maxheight,keepaspectratio}
% Set default figure placement to htbp
\makeatletter
\def\fps@figure{htbp}
\makeatother
\setlength{\emergencystretch}{3em} % prevent overfull lines
\providecommand{\tightlist}{%
  \setlength{\itemsep}{0pt}\setlength{\parskip}{0pt}}
\setcounter{secnumdepth}{-\maxdimen} % remove section numbering
\ifLuaTeX
  \usepackage{selnolig}  % disable illegal ligatures
\fi

\begin{document}
\maketitle

\hypertarget{intro}{%
\section{Intro}\label{intro}}

The World Bank provides valuable data on a number of public health and
economic indicators for countries across the globe\footnote{You can
  access the data at \url{http://data.worldbank.org/}}. Today, we will
be looking indicators which might predict infant mortality, which is the
number of children (per 1000 births) who die before the age of 1.

\subsection*{Questions}
\begin{itemize}
\item What factors do you think might affect or correlate with infant mortality?
\end{itemize}

In particular, we will be looking at 2 specific factors which might
correlate well with infant mortality (measured in 2015) - GDP per capita
(roughly how much income does the average individual produce) as
measured in 2013 and the proportion of the population with access to
electricity (as measured in 2012). I have removed countries which were
missing data for any of the variables.

\textless\textless\textgreater\textgreater= wb.data \textless-
read.csv(``world\_bank\_data.csv'') head(wb.data) @

\subsection*{Questions}
\begin{itemize}
\item What direction do you think the association is between each of these variables?
\item What strength do you think the association is between each of these variables?
\end{itemize}

We can use the \texttt{pairs} command to plot the many pairs of
variables at once. Note that we've excluded the first column here, since
that's just the name of countries \textless\textless fig.height = 6,
fig.width = 6, fig.align = `center'\textgreater\textgreater=
pairs(wb.data{[}, -1{]}) @

\subsection*{Questions}
\begin{itemize}
\item Does this look like what you might expect?
\item What sticks out?
\item Do the relationships look linear?
\end{itemize}

The relationship between electricity and infant mortality looks roughly
linear, but the relationship between GDP per capita and infant mortality
does not. Let's see how we might transform the data. The \texttt{log}
function by default returns the natural log (base e), but we can specify
the base as a second argument. So to take \(\log_10(200)\) we would use
\texttt{log(200, 10)}. Let's plot a few transformations and see what
makes the relationship linear

\textless\textless fig.height = 4, fig.width = 8, fig.align =
`center'\textgreater\textgreater= \# using the par(mfrow = c(r, c)) puts
multiple \# plots together. The plots are arranged so \# that there are
r rows and c columns

par(mfrow = c(1,3))

\hypertarget{first-argument-is-the-x-variable-second-argument-is-the-y-variable}{%
\section{first argument is the X variable, second argument is the Y
variable}\label{first-argument-is-the-x-variable-second-argument-is-the-y-variable}}

\hypertarget{main-specifies-the-title-xlab-specifies-the-x-axis-label}{%
\section{main specifies the title, xlab specifies the x axis
label}\label{main-specifies-the-title-xlab-specifies-the-x-axis-label}}

\hypertarget{and-ylab-specifies-the-y-axis-label}{%
\section{and ylab specifies the y axis
label}\label{and-ylab-specifies-the-y-axis-label}}

plot(wb.data\(gdp_capita, wb.data\)inf\_mort, main = ``Untransformed'',
xlab = ``gdp per capita'', ylab = ``Infant Mortality (per 1000)'')

plot(wb.data\(gdp_capita, log(wb.data\)inf\_mort,10), main =
``log(mortality) \textasciitilde{} gdp/capita'', xlab = ``gdp per
capita'', ylab = ``log(mortality)'')

plot(log(wb.data\(gdp_capita, 10), log(wb.data\)inf\_mort, 10), main =
``log(mortality) \textasciitilde{} log(gdp/capita)'', xlab =
``log(gdp/capita)'', ylab = ``log(mortality)'')

@

\subsection*{Questions}
\begin{itemize}
\item Which transformation looks most linear?
\end{itemize}

The transformation that looks most linear requires taking the log of
both mortality and gdp per capita. This corresponds to a model of
\begin{equation}
\log(\text{mortality}) = a + b\log(\text{gdp/capita}) + \epsilon
\end{equation}

Since we've transformed both variables (not just the y variable), the
interpretation changes slightly. We can interpret the slope parameter
\(b\) as follows- ``Every percentage increase in GDP/Capita, is
associated a b\% increase/decrease in infant mortality.'' Notice now,
that we are talking about percentage changes in the variables of which
we have taken the log.

Let's estimate the two models now using the \texttt{lm} command.

\textless\textless\textgreater\textgreater= \# regression with
electricity as X variable \# note that since we specify the data frame,
we can directly use \# the variable names in the formulat \#
alternatively, we could have used \#
lm(wb.data\(inf_mort ~ wb.data\)elec\_acc) elec.reg \textless-
lm(inf\_mort \textasciitilde{} elec\_acc, data = wb.data)

summary(elec.reg)

\hypertarget{regression-with-electricity-as-x-variable}{%
\section{regression with electricity as X
variable}\label{regression-with-electricity-as-x-variable}}

gdp.reg \textless- lm(log(inf\_mort) \textasciitilde{} log(gdp\_capita),
data = wb.data)

summary(gdp.reg) @

We can also calculate the \(\text{SS}_{total}\)
\(\text{SS}_{regression}\) and \(\text{SS}_{errors}\) for the regression
with electricity accesss. Using these quantities, we can calculate the
\(r^2\) value.\\
\textless\textless\textgreater\textgreater= ss.total \textless-
sum((wb.data\(inf_mort - mean(wb.data\)inf\_mort))\^{}2)

\hypertarget{get-the-estimated-coefficients-from-the-regression}{%
\section{Get the estimated coefficients from the
regression}\label{get-the-estimated-coefficients-from-the-regression}}

\hypertarget{elec.regcoeff-gets-a-vector-the-regression-coefficients-the-first-element-is-the-y-intercept-and-the-second-element-is-the-slope-a.hat---elec.regcoeff1}{%
\section{\texorpdfstring{elec.reg\(coeff gets a vector the regression coefficients # The first element is the y intercept, and the second element is the slope a.hat <- elec.reg\)coeff{[}1{]}}{elec.regcoeff gets a vector the regression coefficients \# The first element is the y intercept, and the second element is the slope a.hat \textless- elec.regcoeff{[}1{]}}}\label{elec.regcoeff-gets-a-vector-the-regression-coefficients-the-first-element-is-the-y-intercept-and-the-second-element-is-the-slope-a.hat---elec.regcoeff1}}

b.hat \textless-
elec.reg\(coeff[2] inf_mort.hat <- a.hat + b.hat * wb.data\)elec\_acc

\hypertarget{calculate-ss_regression}{%
\section{Calculate SS\_regression}\label{calculate-ss_regression}}

ss.regression \textless- sum((inf\_mort.hat -
mean(wb.data\$inf\_mort))\^{}2)

\hypertarget{calculate-ss_error}{%
\section{Calculate SS\_error}\label{calculate-ss_error}}

ss.error \textless- sum((inf\_mort.hat - wb.data\$inf\_mort)\^{}2)

\hypertarget{check-that-ss.regression-ss.error-ss.total}{%
\section{Check that ss.regression + ss.error =
ss.total}\label{check-that-ss.regression-ss.error-ss.total}}

ss.regression + ss.error ss.total

\hypertarget{r2-the-long-way}{%
\section{r\^{}2 the long way}\label{r2-the-long-way}}

ss.regression / ss.total

\hypertarget{r2-the-short-way}{%
\section{r\^{}2 the short way}\label{r2-the-short-way}}

cor(wb.data\(inf_mort, wb.data\)elec\_acc)\^{}2 @

If you look back to the summary of the regression above, you'll notice
that this is the value reported as multiple R squared. Note, that we can
also get the predicted values and residuals directly from the regression
object \textless\textless\textgreater\textgreater= \# residuals resid
\textless- elec.reg\$residuals

\hypertarget{predicted-values-also-called-fitted-values}{%
\section{Predicted values (also called fitted
values)}\label{predicted-values-also-called-fitted-values}}

predicted \textless- elec.reg\$fitted.values

@

\subsubsection*{Questions}
\begin{itemize}
\item Compare the $r^2$ from the gdp regression to the $r^2$ of the electricity regression. What does this suggest about which explanatory variable is a better predictor of infant mortality? 
\item Why do you think this is true?
\item Note that we aren't exactly comparing apples to apples here because one regression has log(mortality) as the response while the other uses mortality untransformed.
\end{itemize}

Let's plot the data and the best fit line for each explanatory variable.
For the gdp plot, we have highlighted Equitorial Guinea in blue which
has a much higher infant mortality rate than we would expect based on
their GDP per capita. \textless\textless fig.width = 8, fig.height = 4,
fig.align = `center'\textgreater\textgreater= par(mfrow = c(1,2))
plot(wb.data\(elec_acc, wb.data\)inf\_mort, main = ``Electricity vs
Infant Mortality'', xlab = ``Electricity Access'', ylab = ``Infant
Mortality'') abline(a = elec.reg\(coeff[1], b = elec.reg\)coeff{[}2{]},
col = ``red'')

plot(log(wb.data\(gdp_capita), log(wb.data\)inf\_mort), main =
``log(GDP/Capita) vs log(Infant Mortality)'', xlab =
``log(GDP/Capita)'', ylab = ``log(Infant Mortality)'') abline(a =
gdp.reg\(coeff[1], b = gdp.reg\)coeff{[}2{]}, col = ``red'')

\hypertarget{the-points-command-can-plot-on-existing-frames}{%
\section{the points command can plot on existing
frames}\label{the-points-command-can-plot-on-existing-frames}}

\hypertarget{col-specifies-the-color-and-pch-specifies-the-shape-of-the-mark}{%
\section{col specifies the color, and pch specifies the shape of the
mark}\label{col-specifies-the-color-and-pch-specifies-the-shape-of-the-mark}}

points(log(wb.data\(gdp_capita)[65], log(wb.data\)inf\_mort){[}65{]},
col = ``blue'', pch = 19) @

\subsubsection*{Questions}
\begin{itemize}
\item According to Wikipedia, Equitorial Guinea has the highest GDP per capita in Sub-Saharan Africa due to significant oil production. However, that wealth is concentrated in a few elites, and not distributed evenly. Instead of GDP per capita (which is a mean), what might be an even better way to predict infant mortality?
\end{itemize}

\begin{Shaded}
\begin{Highlighting}[]
\FunctionTok{summary}\NormalTok{(cars)}
\end{Highlighting}
\end{Shaded}

\begin{verbatim}
##      speed           dist       
##  Min.   : 4.0   Min.   :  2.00  
##  1st Qu.:12.0   1st Qu.: 26.00  
##  Median :15.0   Median : 36.00  
##  Mean   :15.4   Mean   : 42.98  
##  3rd Qu.:19.0   3rd Qu.: 56.00  
##  Max.   :25.0   Max.   :120.00
\end{verbatim}

\hypertarget{including-plots}{%
\subsection{Including Plots}\label{including-plots}}

You can also embed plots, for example:

\includegraphics{lab2_files/figure-latex/pressure-1.pdf}

\end{document}
